\lecture{What is Fair? -- 2/07/24}

Calculating PnL
\begin{itemize}
    \item \vocab{PnL}: Profit and Loss, i.e. how much money you're making
    \item Common scenario: there's a resting offer on the book at $ \$X $ price, someone comes and takes it
    \item I sold the product for $ \$X $ each
    \item If the true value (\vocab{fair}) is $ \$Y $, $ \$(X-Y) $ is your PnL on each trade
    \item If you traded $ n $ contracts, then your PnL is $ \$n \cdot (X-Y) $
\end{itemize}

A \vocab{theoretical fair} is a valuation derived from data / model that you build
Eventually, the box (contract) will reach a \vocab{market price} (for now, we're going to take the midpoint between the best bid and offer)

Let's say the midpoint for the box is \$19000. You think it's \$15600. Who's right? What do you want to do?
\begin{itemize}
    \item Maybe you're kind of confident, but you should take into consideration what the market thinks
    \item For 10 vs 1000 box-traders, the more people trading the more it's likely they are more right (if they're smart)
    \item Basically, your \vocab{internal fair} (what you ``actually believe'') is probably somewhere in between, and many times closer to the market
\end{itemize}

What is... ``Actually'' Fair?
\begin{itemize}
    \item Figuring out the \vocab{true fair} is not that straightforward, but it's what we hope to be able to model
    \item Maybe you have some crystal ball that tells you what the stock price will be at any point is in the future
    \item But really, you never know exactly where the price will go. There's some irriducible noise that you shouldn't expect to predict
\end{itemize}

True Fair 2 Electric Boogaloo

\vocab{True fair}: the expected value of the future price given all the available information

\vocab{Efficient Market Hypothesis (EMH)}: In a highly liquid market like domestic equities, the efficient market hypothesis states that the market price is the true fair
\begin{itemize}
    \item Markets should have already factored in all information that one can know about the stock price
    \item In that case, future moves should be unpredictable, and it should look like a random walk
    \item Everything is 0 EV
    \item This is why index investing is more popular, since stocks generally just go up
\end{itemize}

Edge
\begin{itemize}
    \item What happens if everyone just index invests?
        \begin{itemize}
            \item Well, prices would just rise or flal across the board, wouldn't reflect thoeretical fairs
            \item Someone needs to do \vocab{price discovery}: pushing price back to fair
            \item Let's just call these people ``hedge funds'' (kind of an oversimplification)
        \end{itemize}
    \item \vocab{Edge}: if your fair is closer to the true fair than the market price
    \item Sources of edge:
        \begin{itemize}
            \item \vocab{Analytical}: better at interpreting data and drawing conclusions than the market
            \item Informational: have access to data that others do not
            \item \vocab{Temporal}: move fast, first to react to news
        \end{itemize}
\end{itemize}

You should \textit{constantly} be asking yourself what your edge is. If you don't have a clear edge, you probably shouldn't be trading.

\textbf{Fading: Scenario A}

Market making on AAPL, 99@101. Most of the day you're doing both sides of the trades and making a steady profit. Someone comes in and lifts all the 101 offers. You put out more offers at 101. Someone buys those too. Happens two more times. 

Maybe consider moving your internal fair up.

\textbf{Fading: Scenario B}

Investor buying AAPL. Market 99@101. Buy a lot at 101, market doesn't move, buy more, price doesn't move. Same thing repeats.

Maybe consider moving your internal fair down— market makers appear not to agree with your belief that AAPL is worth more than 101.

When other people have edge and are trading against you, you should probably update your internal fair to incorporate their beliefs. We say that you \vocab{fade} to the market.
\begin{itemize}
    \item How big they bet against you should probably reflects how confident they are
    \item You should probably be thinking about their intentions and who they are (do they actually have edge?)
\end{itemize}
