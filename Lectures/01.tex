\lecture{Introduction -- 1/31/24}

\subsection{Why Trade?}

\subsubsection{Why trade? (Generally)}

\begin{itemize}
    \item Trade: to freely buy or sell a good or service
    \item Motiviation: people trade because it benefits them
    \item In your everday life, you willingly make plenty of trades (ex. buying Chipotle for lunch)
\end{itemize}

\subsubsection{Why trade stocks?}
\begin{itemize}
    \item People tend to associate trading with stocks, and they are certainly traded
    \item Why do they have value?
        \begin{itemize}
            \item Dividends: some stocks pay out money to shareholders
            \item Liquidation: when company goes bankrupt, shareholders have some rights to company's assets
        \end{itemize}
    \item Why trade them though?
        \begin{itemize}
            \item You like a company
            \item Buy and hold for the long term
            \item You have an opinion (on the value of the stock/industry)
            \item Helps you manage risk
            \item The government says you have to (ex you hold too much stock, or to be owner, must own some percent, etc.)
        \end{itemize}
\end{itemize}

\subsection{Financial Products}
\begin{itemize}
    \item \vocab{Bonds}: ways for companies to raise money
    \item Crypto: go brr
    \item Mortgages
    \item Events
    \item Commodoties
\end{itemize}

Liquidity as a Service
\begin{itemize}
    \item \vocab{Liquidity}: the ability to buy and sell on demand
    \item that is, someone needs to be there on the other side of the trades
    \item \vocab{Providers}: people willing to do trades with you
        \begin{itemize}
            \item These providers take on risk, as prices could move]
            \item Providers kind of do good -- without them you literally could not trade
        \end{itemize}
\end{itemize}

What's a Market?
\begin{itemize}
    \item \vocab{Market}: a set of orders to buy or sell a product
    \item Orders exist on a \vocab{order book}
    \item Orders to buy are called \vocab{bids}, orders to sell are called \vocab{offers}
    \item \vocab{Market Makers}: those who set the prices on the market
\end{itemize}

More Terminology
\begin{itemize}
    \item Hit the bid, sold!
    \item Lift the offer, take!
    \item Wanting to buy at 99 dollars at a quantity of 2000 = 99 bid for 2000
    \item Wanting to sell at 101 dollars at a quantity of 1000 = 1000 offered at 101
    \item Markets can \vocab{cross} if a bid is higher than an offer or vice versa
    \item \vocab{Width / Spread}: the difference between the best bid and offer (BBO)
\end{itemize}

Lessons Learned
\begin{itemize}
    \item \vocab{Asymmetric information}: sometimes you know less than others, and you should be scard of this!
    \item \vocab{Efficient markets}: over repeated games, markets generally converge to the right answer
    \item \vocab{Adverse selection}: who is trading against you? Why?
\end{itemize}

