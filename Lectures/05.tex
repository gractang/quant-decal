\lecture{Options -- 2/28/24}

\subsection{What is an Option?}%
\label{sub:What is an Option?}

\begin{itemize}
    \item An option is the right, but not the obligation, to buy or sell a \vocab{contract size} (quantity) of an \vocab{underlying} asset for a specified \vocab{strike} price on a given \vocab{expiry date}.
    \item They can be used to hedge risk, increase leverage, or speculate on the probability distribution of future price movements.
    \item Since options are strictly beneficial for the option holder (you don't have to buy/sell if you don't want to!) it costs a \vocab{premium} to purchase them.
\end{itemize}

\subsection{Why Options?}%
\label{sub:Why Options?}

\begin{itemize}
    \item Options give us exposure to different price levels, allowing us to express complex beliefs about future price distributions
    \item Stocks allow us to make directional bets (long or short shares), but options allow us to make bets on
        \begin{itemize}
            \item stock direction
            \item stock staying within or outside a range within a timeframe
            \item volatility
            \item distributional bets
            \item dispersion bets
        \end{itemize}
\end{itemize}

\subsection{Important Options Terms}%
\label{sub:Important Options Terms}
\begin{itemize}
    \item \vocab{Long}: buying options
    \item \vocab{Short}: selling options
    \item \vocab{Strike Price}: specified price that we transact at in an option
    \item \vocab{Premium}: price that we pay for an option
    \item \vocab{Multiplier}: amount of the asset that we trade in an option (usually, 100 for stock options)
    \item \vocab{Expiration Date}: closing price date
    \item % TODO 
\end{itemize}

\subsection{Pricing Options}%
\label{sub:Pricing Options}
We'll get more into this later, but here's a simpler example to get us thinking about how options might be priced:

\begin{example}
    Three coins are flipped. The price settles at a dollar for every coin that comes up heads. What is the fair price for the \$1 call option?

    \begin{solution}
        Let's consider all the possible settling prices: 0, 1, 2, and 3. The odds of 0 and 3 are identical at $ \left( \frac{1}{2} \right) ^3 = \frac{1}{8} $. By symmetry (or casework), the odds of 1 and 2 are equal at $ \frac{3}{8} $. 
        
        If the price settles at 0 or 1, we don't make any money (recall that we don't \textit{have} to exercise our option, hence the name option). If we bought at 1 when the price is zero that would be bad, so we simply don't buy. 

        If the prices settles at 2 or 3, we make money; specifically, we make a dollar if it's 2 and two dollars if it's 3 (you've bought the stock for 1 but it's actually worth 2 or 3).

        Combining these, we can calculate the expected value of the option: 
        \begin{equation*}
            \frac{1}{8}(0) + \frac{3}{8}(0) + \frac{3}{8}(1) + \frac{1}{8}(2) = \boxed{\frac{5}{8}}.
        \end{equation*}
        
    \end{solution}
    
\end{example}

\subsection{Intrinsic vs Extrinsic Value}%
\label{sub:Intrinsic vs Extrinsic Value}
\begin{itemize}
    \item An option has intrinsic value if you can exercise it immediately for a profit
        \begin{itemize}
            \item For example, if you own the 100-strike Apple call and it's currently trading at 150, then the call must be worth at least 50 (otherwise someone could buy it, exercise it, and sell the stock for a profit)
            \item Thus the option has an \vocab{intrinsic} value of 50
        \end{itemize}
    \item The remainder of the option's price is its \vocab{extrinsic value}, or \vocab{premium}
        \begin{itemize}
            \item The 100-strike call from the earlier example might cost a total of 55, for \$5 of premium.
            \item This represents the actual ``optionality'' you're buying
        \end{itemize}
\end{itemize}

\subsection{Synthetics}%
\label{sub:Synthetics}
\begin{itemize}
    \item A synthetic involves buying a call and selling a put on the same strike
    \item This is usually cheaper than actually buying/selling the stock itself, but with the same payoff
% TODO include payoff diagram later tm
\end{itemize}

\subsection{Put-Call Parity}%
\label{sub:Put-Call Parity}
\begin{itemize}
    \item Buying a call and selling a put has the same payoff structure as buying the stock!
        \begin{itemize}
            \item The value of a stock at the expiry date is the closing price, but the value of the synthetic is the closing price - the strike price
            \item To prevent arbitrage, buying a call, selling a put, and selling the underlying stock must be exactly equal to the futures price
        \end{itemize}
    \item Note that in practice you need to account for interest rates, which is usually done by using the present value of the strike price
    \item Also, there are some technicalities that have to do withthe American vs European options characteristics (in Europe, options can only be exercised on the expiry date; in America, have early-exercise options)
    \item
        \begin{equation*}
            \text{futures price} - \text{call price} + \text{put price} - \text{strike} = 0
        \end{equation*}
        
\end{itemize}

\begin{remark}
    Note that if volatility is high, options tend to worth more (the upside is larger, but you don't lose money if it's bad!)
\end{remark}

\subsection{Greeks}%
\label{sub:Greeks}

\subsubsection{Delta}
\begin{example}
    \begin{itemize}
        \item Apple is currently trading at \$150, and I own the 100-strike call and the 200-strike call. Which option has more delta?

            \begin{solution}
                There's a very low chance the stock drops below \$100, so that call is very likely in the money and getting exercised. However, it's very unlikely that the stock will be above \$200, so that call is probably not in the money. Thus, the 100-strike call has more delta.
            \end{solution}
            
    \end{itemize}
\end{example}

