\RequirePackage{tikz}
\RequirePackage{xspace}
\RequirePackage{pgfplots}
\usepgfplotslibrary{polar}

%%% EECS126 Commonly Used Symbols %%%
\newcommand{\PP}{\mathbb{P}} % probability
\newcommand{\EE}{\mathbb{E}} % expectation
\newcommand{\LL}{\mathbb{L}} % likelihood
\newcommand{\G}{\mathcal{G}} % for erdos-renyi

\renewcommand{\iff}{\Leftrightarrow}
\renewcommand{\implies}{\Rightarrow}

%%% Other Useful Math Terms %%%
\newcommand{\VV}{\mathbb{V}}
\newcommand{\U}{\mathbb{U}}
\newcommand{\ZZ}{\mathbb{Z}} % integers
\newcommand{\RR}{\mathbb{R}} % reals
\newcommand{\QQ}{\mathbb{Q}} % rationals
\newcommand{\CC}{\mathbb{C}} % complex numbers
\newcommand{\FF}{\mathbb{F}} % arbitrary field
\newcommand{\NN}{\mathbb{N}} % natural numbers

\newcommand{\cbrt}[1]{\sqrt[3]{#1}}
\newcommand{\floor}[1]{\left\lfloor #1 \right\rfloor}
\newcommand{\ceil}[1]{\left\lceil #1 \right\rceil}
\newcommand{\mailto}[1]{\href{mailto:#1}{\texttt{#1}}}
\newcommand{\ol}{\overline}
\newcommand{\ul}{\underline}
\newcommand{\wt}{\widetilde}
\newcommand{\wh}{\widehat}
\newcommand{\eps}{\varepsilon}
\newcommand{\vocab}[1]{\textbf{\color{teal} #1}}
\newcommand{\catname}{\mathsf}
\newcommand{\hrulebar}{
  \par\hspace{\fill}\rule{0.95\linewidth}{.7pt}\hspace{\fill}
  \par\nointerlineskip \vspace{\baselineskip}
}
\newcommand{\half}{\frac{1}{2}}

%More commands and math operators
\DeclareMathOperator{\cis}{cis}
\DeclareMathOperator*{\lcm}{lcm}
\DeclareMathOperator*{\argmin}{arg min}
\DeclareMathOperator*{\argmax}{arg max}

%Convenient Environments
\newenvironment{soln}{\begin{proof}[Solution]}{\end{proof}}
\newenvironment{parlist}{\begin{inparaenum}[(i)]}{\end{inparaenum}}
\newenvironment{gobble}{\setbox\z@\vbox\bgroup}{\egroup}

%Inequalities
\newcommand{\cycsum}{\sum_{\mathrm{cyc}}}
\newcommand{\symsum}{\sum_{\mathrm{sym}}}
\newcommand{\cycprod}{\prod_{\mathrm{cyc}}}
\newcommand{\symprod}{\prod_{\mathrm{sym}}}

%From H113 "Introduction to Abstract Algebra" at UC Berkeley
\newcommand{\charin}{\text{ char }}
\DeclareMathOperator{\sign}{sign}
\DeclareMathOperator{\Aut}{Aut}
\DeclareMathOperator{\Inn}{Inn}
\DeclareMathOperator{\Syl}{Syl}
\DeclareMathOperator{\Gal}{Gal}
\DeclareMathOperator{\GL}{GL} % General linear group
\DeclareMathOperator{\SL}{SL} % Special linear group

\DeclareMathOperator{\coker}{coker}
\DeclareMathOperator{\Span}{span}
\DeclareMathOperator{\ind}{ind}

%From Kiran Kedlaya's "Geometry Unbound"
\newcommand{\dang}{\measuredangle} %% Directed angle
\newcommand{\ray}[1]{\overrightarrow{#1}}
\newcommand{\seg}[1]{\overline{#1}}
\newcommand{\arc}[1]{\wideparen{#1}}

%From M275 "Topology" at SJSU
\newcommand{\id}{\mathrm{id}}
\newcommand{\taking}[1]{\xrightarrow{#1}}
\newcommand{\inv}{^{-1}}

%From M170 "Introduction to Graph Theory" at SJSU
\DeclareMathOperator{\diam}{diam}
\DeclareMathOperator{\ord}{ord}
%\newcommand{\defeq}{\overset{\mathrm{def}}{=}}
\newcommand{\defeq}{\coloneqq}

%From the USAMO .tex files
\newcommand{\st}{^{\text{st}}}
\newcommand{\nd}{^{\text{nd}}}
\renewcommand{\th}{^{\text{th}}}
\newcommand{\dg}{^\circ}
\newcommand{\ii}{\item}

% From Math 55 and Math 145 at Harvard
\newenvironment{subproof}[1][Proof]{%
\begin{proof}[#1] \renewcommand{\qedsymbol}{$\blacksquare$}}%
{\end{proof}}

\newcommand{\liff}{\leftrightarrow}
\newcommand{\lthen}{\rightarrow}
\newcommand{\opname}{\operatorname}
\newcommand{\surjto}{\twoheadrightarrow}
\newcommand{\injto}{\hookrightarrow}
\newcommand{\On}{\mathrm{On}} % ordinals
\DeclareMathOperator{\img}{im} % Image
\DeclareMathOperator{\Img}{Im} % Image
\DeclareMathOperator{\Coker}{Coker} % Cokernel
\DeclareMathOperator{\Ker}{Ker} % Kernel
\DeclareMathOperator{\rank}{rank}
\DeclareMathOperator{\Spec}{Spec} % spectrum
\DeclareMathOperator{\Tr}{Tr} % trace
\DeclareMathOperator{\pr}{pr} % projection
\DeclareMathOperator{\ext}{ext} % extension
\DeclareMathOperator{\pred}{pred} % predecessor
\DeclareMathOperator{\dom}{dom} % domain
\DeclareMathOperator{\ran}{ran} % range
\DeclareMathOperator{\Hom}{Hom} % homomorphism
\DeclareMathOperator{\End}{End} % endomorphism
\DeclareMathOperator{\ch}{ch} % characteristic

% Things Lie
\newcommand{\kg}{\mathfrak g}
\newcommand{\kh}{\mathfrak h}
\newcommand{\kn}{\mathfrak n}
\newcommand{\ku}{\mathfrak u}
\newcommand{\kz}{\mathfrak z}
\DeclareMathOperator{\Ext}{Ext} % Ext functor
\DeclareMathOperator{\Tor}{Tor} % Tor functor

% More script letters etc.
\newcommand{\SA}{\mathscr A}
\newcommand{\SB}{\mathscr B}
\newcommand{\SC}{\mathscr C}
\newcommand{\SD}{\mathscr D}
\newcommand{\SF}{\mathscr F}
\newcommand{\SG}{\mathscr G}
\newcommand{\SH}{\mathscr H}
\newcommand{\OO}{\mathcal O}


%%fakesection Napkin Macros
\newcommand{\Zc}[1]{\ZZ/#1\ZZ}
\newcommand{\Zcc}[1]{\ZZ/(#1)\ZZ}
\newcommand{\Zm}[1]{(\ZZ/#1\ZZ)^\times}
\newcommand{\TT}{\mathbb T}
\newcommand{\im}{^{\text{img}}} % or ``?
\newcommand{\pre}{^{\text{pre}}}
\newcommand{\normalin}{\trianglelefteq}
\newcommand{\triv}{\mathrm{triv}}
\newcommand{\largeotimes}[1]{\mathop{\otimes}\limits_{#1}}
\newcommand{\Mat}{\mathrm{Mat}}
\newcommand{\PGL}[1]{\mathbf{PGL}_{#1}(\CC)}

\DeclareMathOperator{\Stab}{Stab}
\DeclareMathOperator{\FixPt}{FixPt}
\DeclareMathOperator{\refl}{refl}
\DeclareMathOperator{\Fun}{Fun}
\DeclareMathOperator{\Irrep}{Irrep}
\DeclareMathOperator{\Res}{Res}
\DeclareMathOperator{\Reg}{Reg}
\DeclareMathOperator{\Classes}{Classes}
\DeclareMathOperator{\Frac}{Frac}
\newcommand{\FunCl}{\Fun_{\mathrm{class}}}
\newcommand{\Homrep}{\Hom_{\mathrm{rep}}}
\newcommand{\ab}{^\text{ab}} % abelianization
\DeclareMathOperator{\ev}{ev}
\DeclareMathOperator{\Wind}{\mathbf I}
\newcommand{\Ctriv}{\CC_{\mathrm{triv}}}
\newcommand{\Csign}{\CC_{\mathrm{sign}}}

% Alg geom macros
\newcommand{\V}{\mathcal V}
\newcommand{\Vp}{\mathcal V_{\text{pr}}}
\newcommand{\Aff}{\mathbb A}
\newcommand{\II}{\mathcal I_{\text{rad}}}
\newcommand{\RP}{\mathbb{RP}}
\newcommand{\CP}{\mathbb{CP}}
\newcommand{\Cl}{\mathrm{Cl}}
\newcommand{\restrict}[1]{\bgroup\restriction_{#1}\egroup}
\DeclareMathOperator{\Opens}{OpenSets}
\DeclareMathOperator{\Proj}{Proj}
\DeclareMathOperator{\res}{res}
\newcommand{\km}{\mathfrak m}
\newcommand{\sh}{^\mathrm{sh}}

\renewcommand{\Re}{\opname{Re}}
\renewcommand{\Im}{\opname{Im}}

%% Category Theory macros
\renewcommand{\AA}{\mathcal A}
\newcommand{\BB}{\mathcal B}
\newcommand{\obj}{\operatorname{obj}}
\newcommand{\op}{^{\mathrm{op}}}


\usepackage{tkz-graph}
\pgfarrowsdeclare{biggertip}{biggertip}{%
  \setlength{\arrowsize}{1pt}
  \addtolength{\arrowsize}{.1\pgflinewidth}
  \pgfarrowsrightextend{0}
  \pgfarrowsleftextend{-5\arrowsize}
}{%
  \setlength{\arrowsize}{1pt}
  \addtolength{\arrowsize}{.1\pgflinewidth}
  \pgfpathmoveto{\pgfpoint{-5\arrowsize}{4\arrowsize}}
  \pgfpathlineto{\pgfpointorigin}
  \pgfpathlineto{\pgfpoint{-5\arrowsize}{-4\arrowsize}}
  \pgfusepathqstroke
}
\tikzset{
	EdgeStyle/.style = {>=biggertip}
}

\usepackage[all,cmtip,2cell]{xy}
\UseTwocells
\newcommand\nattfm[5]{\xymatrix@C+2pc{#1 \rtwocell<4>^{#2}_{#4}{\; #3} & #5}}

%% Alg Top macros
\DeclareMathOperator{\Cells}{Cells}
\newcommand{\HdR}{H_{\mathrm{dR}}}

%% Alg NT macros
\newcommand{\ka}{\mathfrak a}
\newcommand{\kb}{\mathfrak b}
\newcommand{\kp}{\mathfrak p}
\newcommand{\kq}{\mathfrak q}

\newcommand{\Frob}{\mathrm{Frob}}
\DeclareMathOperator{\Norm}{N}
\DeclareMathOperator{\Ram}{Ram}
\newcommand{\TrK}{\Tr_{K/\QQ}}
\newcommand{\NK}{\Norm_{K/\QQ}}
\newcommand{\kf}{\mathfrak f}
\newcommand{\kP}{\mathfrak P}
\newcommand{\kQ}{\mathfrak Q}

%% Diff geo macros
\newcommand{\ee}{\mathbf e}
\newcommand{\fpartial}[2]{\frac{\partial #1}{\partial #2}}

%% ST macros
\newcommand{\CH}{\mathsf{CH}}
\newcommand{\ZFC}{\mathsf{ZFC}}

\newcommand{\Name}{\text{Name}}
\newcommand{\Po}{\mathbb P}
\newcommand{\nrank}{\opname{n-rank}} % ranks

\newcommand{\EmptySet}{\mathrm{EmptySet}}
\newcommand{\PowerSet}{\mathrm{PowerSet}}
\newcommand{\Pairing}{\mathrm{Pairing}}
\newcommand{\Infinity}{\mathrm{Infinity}}
\newcommand{\Extensionality}{\mathrm{Extensionality}}
\newcommand{\Foundation}{\mathrm{Foundation}}
\newcommand{\Union}{\mathrm{Union}}
\newcommand{\Comprehension}{\mathrm{Comprehension}}
\newcommand{\Replacement}{\mathrm{Replacement}}

\usepackage{mathrsfs}

\newcommand\MM{\mathscr M}
\newcommand\llex{<_{\text{lex}}}

\DeclareMathOperator{\cof}{cof}

%% Quantum macros
\usepackage{braket}
\newcommand{\cvec}[1]{\begin{bmatrix} #1 \end{bmatrix}}
\newcommand{\pair}[2]{\begin{bmatrix} #1 \\ #2 \end{bmatrix}}
\newcommand{\zup}{\ket\uparrow}
\newcommand{\zdown}{\ket\downarrow}
\newcommand{\xup}{\ket\rightarrow}
\newcommand{\xdown}{\ket\leftarrow}
\newcommand{\yup}{\ket\otimes}
\newcommand{\ydown}{\ket\odot}
\newcommand{\UCNOT}{U_{\mathrm{CNOT}}}
\newcommand{\UQFT}{U_{\mathrm{QFT}}}

%% Measure
\DeclareMathOperator{\Var}{Var}
\newcommand{\cme}{^\text{cm}}
\newcommand{\asto}{\xrightarrow{\text{a.s.}}}

%% Hot chili peppers
\reversemarginpar
\newcommand{\prechili}{\vspace*{0.3em}\hspace*{1.5em}}
\newcommand{\nochili}{\hspace*{1.5em}}
\newcommand{\chili}{\includegraphics[width=1.5em]{media/chili.png}}
\newcommand{\gim}{\marginpar{\prechili\nochili\nochili\chili}}
\newcommand{\yod}{\marginpar{\prechili\nochili\chili\chili}}
\newcommand{\kurumi}{\marginpar{\prechili\chili\chili\chili}}

\newcommand{\poisson}{\mathbin{\alpha\kern-0.5em{\cdot}\:}} % for the fish symbol
\newcommand{\person}{\resizebox{!}{2.5ex}{\usebox{\stickfiguretikz}}\xspace} % for the person symbol
\newcommand{\flower}{\resizebox{1em}{!}{\usebox{\flowerbox}}\xspace} % for the flower symbol

% Problems
\newcommand{\UNSAT}{\textsc{UNSAT}}
\newcommand{\SAT}{\textsc{SAT}}
\newcommand{\STCONN}{\textsc{stconn}}
\newcommand{\OPSTCONN}{\textsc{Optimal-stconn}}
\newcommand{\STRCONN}{\textsc{Strong-Conn}}
\newcommand{\CHR}{\textsc{Chromatic\#4}}
\newcommand{\CLQ}{\textsc{CLIQUE}}
\newcommand{\COL}{\textsc{3COL}}

% Classes
% \renewcommand{\P}{\textsf{P}}
\newcommand{\NP}{\textsf{NP}}
\renewcommand{\L}{\textsf{L}}
\newcommand{\NL}{\textsf{NL}}
\newcommand{\coNP}{\textsf{coNP}}
\newcommand{\coNL}{\textsf{coNL}}
\newcommand{\MAJPP}{\textsf{MAJPP}}
\newcommand{\coRP}{\textsf{coRP}}
\newcommand{\ZPP}{\textsf{ZPP}}
\newcommand{\PPoly}{\textsf{P/poly}}
\newcommand{\BPP}{\textsf{BPP}}
\newcommand{\IP}{\textsf{IP}}
\newcommand{\PSPACE}{\textsf{PSPACE}}
\newcommand{\TIME}{\textsf{TIME}}
\newcommand{\SPACE}{\textsf{SPACE}}

% these are custom commands and symbols specific to my pset template
\NeedsTeXFormat{LaTeX2e}
\ProvidesPackage{symbolsandcommands}[symbols and commands for personal use]
\RequirePackage{tikz}
\RequirePackage{xspace}
\RequirePackage{pgfplots}
\usepgfplotslibrary{polar}
\pgfplotsset{compat=newest}

% creating the person stick figure
\newsavebox{\stickfiguretikz}
\sbox{\stickfiguretikz}{%
\begin{tikzpicture}[baseline= 2ex]
    \tikzstyle{head} = [circle, draw, scale=2, ultra thick]
	\node [style=head] (0) at (0, 3) {};
	\node (1) at (-0.75, 1.55) {};
	\node (2) at (0.75, 1.55) {};
	\node (3) at (-0.425, 2.1) {};
	\node (4) at (0.425, 2.1) {};
	\node (5) at (-0.4, 0) {};
	\node (6) at (0.4, 0) {};
	\node (7) at (0, 1.5) {};
	\draw [in=90, out=-165, ultra thick] (0.south) to (1.center);
	\draw [in=90, out=-15, ultra thick] (0.south) to (2.center);
	\draw [bend left=5, ultra thick] (1.center) to (3.center);
	\draw [bend right=5, ultra thick] (2.center) to (4.center);
	\draw [bend left=5, ultra thick] (3.center) to (5.center);
	\draw [bend right=5, ultra thick] (4.center) to (6.center);
	\draw [ultra thick] (5.center) to (7.center);
	\draw [ultra thick] (7.center) to (6.center);
\end{tikzpicture}%
}

\definecolor{stemgreen}{RGB}{79, 207, 14}
\definecolor{flowerpurple}{RGB}{200, 145, 245}
\newsavebox{\flowerbox}
\sbox{\flowerbox}{%
  \begin{tikzpicture}[x=1ex, y=1ex, baseline=5ex]
    \begin{scope}[scale=4, shift={(4.95,-0.55)}]
      \node (0) at (0, -5) {};
      \node (1) at (4.25, 1.75) {};
      \node (2) at (0.75, 6.25) {};
      \draw [in=90, out=150, color=stemgreen, line width=0.8mm] (1.center) to (0.center);
      \draw [in=-120, out=-105, looseness=2.00, color=stemgreen, line width=0.8mm] (2.center) to (1.center);
    \end{scope}
      \begin{polaraxis}[grid=none, axis lines=none]
        \addplot[mark=none,domain=0:360,samples=300,color=flowerpurple, line width=0.8mm, fill=white] {cos(x*4-70)};
      \end{polaraxis}
  \end{tikzpicture}%
}

\DeclareMathOperator{\spn}{span} % span
\DeclareMathOperator{\adj}{adj} % adjugate/adjoint
\DeclareMathOperator{\vol}{Vol} % volume
\DeclareMathOperator{\var}{Var} % variance
\DeclareMathOperator{\stab}{Stab} % stabilizer
\DeclareMathOperator{\conj}{conj} % conjugation

\newcommand{\limty}[1]{\lim_{#1 \to \infty}} % limit to infinity. Usage: \limty{n}

\newcommand{\subt}[2]{#1_{\text{#2}}} % text subscript. Usage: \subt{a}{text}
\newcommand{\supt}[2]{#1^{\text{#2}}} % text superscript. Usage: \supt{a}{text}

\newcommand{\dspace}{\,}
\newcommand{\dstyle}{\text} 
\newcommand{\dx}{\dspace{\dstyle d}x}

%%% weird 154 things %%%
\newcommand{\setbld}[2]{\ensuremath{\left\{#1 ~\middle|~ #2\right\}}}
\def\B{\ensuremath{\left\{0, 1\right\}}}
\newcommand{\setbldn}[2]{\ensuremath{\{#1 ~|~ #2\}}}  
\def\BigO{\ensuremath{\mathcal{O}}}
\def\ATM{\ensuremath{\textsc{A}_\textsc{TM}}}
\newcommand{\repr}[1]{\ensuremath{\langle{#1}\rangle}}
\def\NP{\ensuremath{\textsf{NP}}}