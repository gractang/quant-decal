\usepackage[usenames,svgnames,dvipsnames,table]{xcolor}
\usepackage[margin = 1in]{geometry} % package for setting the margins
\usepackage{amsmath, amssymb, amsthm, url} % packages for almost everything math-related
\usepackage{mathtools}
\usepackage{problems} % for the problems
\usepackage{wasysym} % package for some nice symbols like \smiley
\usepackage{tikz} % package for drawing stuff
\usepackage{tikzsymbols} % package for the \person command
\usepackage{fancyhdr} % package for the header formatting
\usepackage{enumitem} % package 
\usepackage{commath} % package for things like \del, \cbr, and \sbr. These handle parentheses well.
\usepackage{graphicx} % package for inserting images
% TODO: figure out minted
\usepackage{tcolorbox}
\usepackage{mathrsfs}
\usepackage{textcomp}
\usepackage[textsize=scriptsize,shadow]{todonotes}
\usepackage{mathtools}
\usepackage{microtype}
\usepackage[normalem]{ulem}
\usepackage{stmaryrd}
\usepackage{wasysym}
\usepackage{multirow}
\usepackage{prerex}
\usepackage{hyperref}
\usepackage[nameinlink]{cleveref}
\usepackage[framemethod=TikZ]{mdframed}
\usepackage{thmtools}
\usepackage{tikz-cd}
\usepackage{pgfplots}
\usepackage{bbm} % for indicator variable
\pgfplotsset{compat=newest}
\usetikzlibrary{decorations.markings}

\usetikzlibrary{patterns}
\usepackage{tikz-3dplot}

\usepackage[utf8]{inputenc}
\usepackage{color}


%%fakesection evan.sty macros
%Small commands
%% Napkin commands
\newcommand{\prototype}[1]{
	\emph{{\color{red} Prototypical example for this section:} #1} \par\medskip
}
\newenvironment{moral}{%
	\begin{mdframed}[linecolor=green!70!black]%
	\bfseries\color{green!50!black}}%
	{\end{mdframed}}

%%fakesection Links
\hypersetup{
    colorlinks,
    linkcolor={violet!50!black},
    citecolor={green!50!black},
    urlcolor={blue!80!black}
}

%% HEADERS AND PAGE NUMBERS %%%
\pagestyle{fancy} % for the header
\fancyhf{} % for the header
\lfoot{\thepage} % for the page numbers. change to cfoot for centered numbers
\newcommand{\settitle}[1]{
\begin{center}
    \Large{\textsc{#1}}
\end{center}
}

% \theoremstyle{definition}

% The blue box around Theorems
\mdfdefinestyle{mdbluebox}{%
	roundcorner = 8pt,
	linewidth=1pt,
	skipabove=12pt,
	innerbottommargin=9pt,
	skipbelow=2pt,
	nobreak=true,
	linecolor=NavyBlue!65,
	backgroundcolor=TealBlue!5,
}
\declaretheoremstyle[
	headfont=\bfseries\color{MidnightBlue},
	mdframed={style=mdbluebox},
	headpunct={\\[3pt]},
	postheadspace={0pt}
]{thmbluebox}

\mdfdefinestyle{mdredbox}{%
	linewidth=0.5pt,
	skipabove=12pt,
	frametitleaboveskip=5pt,
	frametitlebelowskip=0pt,
	skipbelow=2pt,
	frametitlefont=\bfseries,
	innertopmargin=4pt,
	innerbottommargin=8pt,
	linecolor=RawSienna,
	backgroundcolor=Salmon!5,
}
\declaretheoremstyle[
	headfont=\bfseries\color{RawSienna},
	mdframed={style=mdredbox},
	headpunct={\\[3pt]},
	postheadspace={0pt},
]{thmredbox}

\declaretheorem[%
style=thmbluebox,name=Theorem,numberwithin=section]{theorem}
\declaretheorem[style=thmbluebox,name=Lemma,sibling=theorem]{lemma}
\declaretheorem[style=thmbluebox,name=Proposition,sibling=theorem]{proposition}
\declaretheorem[style=thmbluebox,name=Corollary,sibling=theorem]{corollary}
\declaretheorem[style=thmredbox,name=Example,sibling=theorem]{example}

\mdfdefinestyle{mdgreenbox}{%
	nobreak=true,
	skipabove=8pt,
	linewidth=2pt,
	rightline=false,
	leftline=true,
	topline=false,
	bottomline=false,
	linecolor=ForestGreen,
	backgroundcolor=ForestGreen!5,
}
\declaretheoremstyle[
	headfont=\bfseries\color{ForestGreen!70!black},
	bodyfont=\normalfont,
	spaceabove=2pt,
	spacebelow=1pt,
	mdframed={style=mdgreenbox},
	headpunct={. },
]{thmgreenbox}
\declaretheoremstyle[
	headfont=\bfseries\sffamily\color{ForestGreen!70!black},
	bodyfont=\normalfont,
	spaceabove=2pt,
	spacebelow=1pt,
	mdframed={style=mdgreenbox},
	headpunct={},
]{thmgreenbox*}

\mdfdefinestyle{mdblackbox}{%
	skipabove=8pt,
	linewidth=3pt,
	rightline=false,
	leftline=true,
	topline=false,
	bottomline=false,
	linecolor=black,
	backgroundcolor=RedViolet!5!gray!5,
}
\declaretheoremstyle[
	headfont=\bfseries,
	bodyfont=\normalfont\small,
	spaceabove=0pt,
	spacebelow=0pt,
	mdframed={style=mdblackbox}
]{thmblackbox}

\theoremstyle{theorem}
\declaretheorem[name=Question,sibling=theorem,style=thmblackbox]{question}
\declaretheorem[name=Exercise,sibling=theorem,style=thmblackbox]{exercise}
\declaretheorem[name=Remark,sibling=theorem,style=thmgreenbox]{remark}
\declaretheorem[name=Remark,sibling=theorem,style=thmgreenbox*]{remark*}
\declaretheorem[name=Step,style=thmgreenbox]{step} % only used in Lebesgue int
\declaretheorem[name=Definition,sibling=theorem,style=thmgreenbox]{definition}

% \theoremstyle{definition}
% \newtheorem{definition}[theorem]{Definition}
\newtheorem{claim}[theorem]{Claim}
\newtheorem{fact}[theorem]{Fact}
\newtheorem{abuse}[theorem]{Abuse of Notation}
\newcommand{\listhack}{$\empty$\vspace{-2em}}


\newenvironment{proofsketch}{%
\renewcommand{\proofname}{Proof Sketch}\proof}{\endproof}